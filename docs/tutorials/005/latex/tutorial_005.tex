\documentclass[a4]{article}
\usepackage{geometry}
\geometry{verbose,tmargin=2.5cm,bmargin=2.5cm,lmargin=3cm,rmargin=3cm}
\usepackage{amsmath,amssymb,amsthm}
\usepackage{graphicx}
\graphicspath{{graphics/}}
\usepackage[utf8]{inputenc}
\usepackage{fancyvrb}
\usepackage{hyperref}
\usepackage{lscape}
\usepackage{adjustbox}
\usepackage{verbatim}

\title{MultiFEBE \\ Tutorial 5: harmonic analysis of an elastic straight beam with finite elements}
\author{\'A.G. Vega-Artiles}
\date{September 2022}

\begin{document}

\maketitle

\begin{figure}[h]
	\centering
	\includegraphics{beam.eps}
	\caption{Problem layout.}
	\label{fig:beam}
\end{figure}

\section{Problem description}

In this fifth tutorial, a static analysis of an elastic straight beam is performed using the Finite Element Method (FEM), i.e. finite elements. Figure \ref{fig:beam} shows the geometry. Required material and geometric properties are the Young's modulus $E$, the Poisson's ratio $\nu$, the density $\rho$, the hysteretic damping $\xi$, the length $L$, the square-shaped cross section $1 \medspace \mathrm{m} \medspace \mathrm{x} \medspace 1 \medspace \mathrm{m}$ and the harmonic force $F$. The force $F$ can be regarded as a P-wave. Self-weight is not considered. 

The natural frequencies to this problem can be obtained by means of the following equations \cite{clough}:

\begin{equation}
	\begin{array}{l}
		\omega_1 = (1.875)^2 \cdot \sqrt{EI /(\overline{m}L^4)} \medspace \mathrm{rad/s} \\
		\omega_2 = (4.694)^2 \cdot \sqrt{EI /(\overline{m}L^4)} \medspace \mathrm{rad/s} \\
		\omega_3 = (7.855)^2 \cdot \sqrt{EI /(\overline{m}L^4)} \medspace \mathrm{rad/s}
	\end{array}
\end{equation}

The problem is solved for $L=10$ $\mathrm{m}$, $E=200\cdot 10^9$ $\mathrm{N/m^2}$, $\nu=0.26$, $\rho=7850$ $\mathrm{kg/m^3}$, mass per unit length $\overline{m}=7850$ $\mathrm{kg/m}$, $\xi=0.02$ and $F=1000$ $\mathrm{N/m^2}$. First, second and third natural frequencies are: 

\begin{equation}
	\begin{array}{l}
		\omega_1 = 51.23 \medspace \mathrm{rad/s} \\
		\omega_2 = 321.05 \medspace \mathrm{rad/s} \\
		\omega_3 = 899.05 \medspace \mathrm{rad/s}
	\end{array}
\end{equation}

\section{Pre-processing} 
Pre-processing in MultiFEBE consists of defining the geometry and mesh of the problem. There are three ways to do such definition: directly from the input file (mode = 0), from another file in the native format (mode = 1) or from another file in the Gmsh MSH file format version 2.2 (mode = 2).

In the current example the problem definition will be read from the input file.       

\section{Solving}
Solving in MultiFEBE consists of running the software by specifying several options in the following sections\footnote{See reference manual.}: [problem], [settings], [materials], [boundaries], [regions] and [conditions over the boundaries].

The first part to configurate is the problem definition in the section [problem]. This example is a 3D harmonic mechanical problem.  

\begin{Verbatim}	
[problem]
type = mechanics
analysis = harmonic
n = 3D
\end{Verbatim}

Then, a list of frequencies is generated by specifying the number of frequencies, that must be $\geq 2$, (10000), followed by the minimum frequency, $>$ 0, ($0.001$) and the maximum frequency ($1000$), being each one in new lines.

\begin{Verbatim}
[frequencies]
rad/s
lin
10000
0.001
1000
\end{Verbatim}

As the problem has just one material, the section [materials] will need two lines: a first line for the number of materials in the model and a second line for the properties such as tag, type, $\rho$, E, $\nu$ and $\xi$.

\begin{Verbatim}
[materials]
1
1 elastic_solid rho 7850. E 200.E9 nu 0.26 xi 0.02
\end{Verbatim}

Next step is to define the mesh. Mesh definition requires three sections: [parts], [nodes] and [elements]. 

In the section [parts], groups with a similar significance are defined. The first line specifies the number of parts (1) and a line per part by indicating the part identifier (1) and the part name (resorte).    

The nodes are defined in the section [nodes]. The first line specifies the number of nodes (5) and a line per node by indicating the node identifier and its coordinates (x y z).    

In the section [elements], all the elements of the model are defined. The first line indicates the number of elements (4) and a line per element indicating the element identifier, the type of element (line2 = line element with 2 nodes), the number of auxiliary tags (1), the part identifier (1) and a list of identifiers corresponding to the nodes of the element.

\begin{Verbatim}	
[parts]
1
1 beam
\end{Verbatim}

\begin{Verbatim}	
[nodes]
5
1 0. 0. 0.
3 2.5 0. 0.
4 5.0 0. 0.
5 7.5 0. 0.
2 10. 0. 0.
\end{Verbatim}

\begin{Verbatim}	
[elements]
4
1 line2 1 1 1 3
2 line2 1 1 3 4
3 line2 1 1 4 5
4 line2 1 1 5 2
\end{Verbatim}

The section [fe subregions] indicates the number of fe subregions in the first line (1) and a line per subregion indicating the subregion identifier (1) and the part identifier (1). The last two zeros at the end are mandatory, and they are going to be used in the future for additional features.

\begin{Verbatim}
[fe subregions]
1
1 1 0 0
\end{Verbatim}

In the section [cross sections], it is necessary to specify the number of cross sections in the first line and a line per cross section by indicating the type of fe (strbeam\_eb = straight beam, Euler–Bernoulli model), number of fe subregions related to the cross section (1), fe subregion identifier (1), type of cross section (rectangle), y dimension (1), z dimension (1), reference vector for y' direction (0. 1. 0.).

\begin{Verbatim}
[cross sections]
1
strbeam_eb 1 1 rectangle 1. 1. 0. 1. 0.
\end{Verbatim}

The format of the [regions] section consists of a first line indicating the number of regions (1). Furthermore, for each region there must be a block of data consisting of several lines of data. The first one is the region identifier and the region class (discretization method) (1 fe). As the region is a FE region, then the second line indicates the number of subregions (1) and its identifier (1).

\begin{Verbatim}	
[regions]
1
1 fe
1 1
material 1
\end{Verbatim}

In the section [conditions over nodes] all boundary conditions over nodes will be specified. As a 3D model, there are 6 lines for every boundary condition. Every line has two digits, where the first one indicates the type of condition (0 for displacement and 1 for force) and the second one the value in complex number (real part, imaginary part) because it is a harmonic analysis. In case of displacement, firstly the displacements $u_x, u_y, u_z$ and then the rotations $\theta_x, \theta_y, \theta_z$. In case of force, firstly the forces $F_x, F_y, F_z$ and then the moments $M_x, M_y, M_z$. 

\begin{Verbatim}	
[conditions over nodes]
node 1: 0 (0.,0.)
        0 (0.,0.)
        0 (0.,0.)
        0 (0.,0.)
        0 (0.,0.)
        0 (0.,0.)

node 2: 1 (0.,0.)
        1 (0.,0.)
        1 (1000.,0.)
        1 (0.,0.)
        1 (0.,0.)
        1 (0.,0.)
\end{Verbatim}

The whole data file applied to the problem is the following:

\begin{Verbatim}
[problem]
type = mechanics
analysis = harmonic
n = 3D

[frequencies]
rad/s
lin
10000
0.001
1000

[materials]
1
1 elastic_solid rho 7850. E 200.E9 nu 0.26 xi 0.02

[parts]
1
1 beam

[nodes]
5
1 0. 0. 0.
3 2.5 0. 0.
4 5.0 0. 0.
5 7.5 0. 0.
2 10. 0. 0.

[elements]
4
1 line2 1 1 1 3
2 line2 1 1 3 4
3 line2 1 1 4 5
4 line2 1 1 5 2

[fe subregions]
1
1 1 0 0

[cross sections]
1
strbeam_eb 1 1 rectangle 1. 1. 0. 1. 0.

[regions]
1
1 fe
1 1
material 1

[conditions over nodes]
node 1: 0 (0.,0.)
        0 (0.,0.)
        0 (0.,0.)
        0 (0.,0.)
        0 (0.,0.)
        0 (0.,0.)

node 2: 1 (0.,0.)
        1 (0.,0.)
        1 (1000.,0.)
        1 (0.,0.)
        1 (0.,0.)
        1 (0.,0.)
\end{Verbatim}

\section{Post-processing}

\subsection{Nodal solutions file (*.nso)}
The results of the node 2 were taken and plotted together with the analytical solution to observe the frequency response in Figure \ref{fig:beam_results}.

It can be seen that the numerical solution is in agreement with the analytical solution but with small discrepancies due to the fact that there are not exact solutions for this problem.

\begin{figure}[h]
	\centering
	\includegraphics[scale = 0.5]{uz_f.eps}
	\caption{Results of the harmonic straight beam.}
	\label{fig:beam_results}
\end{figure}

\begin{thebibliography}{99}
	
	\bibitem{clough} Clough, R. W., and J. Penzien. Dynamics of structures. Computers \& Structures, Inc., (2003).
	
\end{thebibliography}

\end{document}
